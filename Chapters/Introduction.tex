\chapter{Introduction}
In this thesis, we will examine several cybersecurity vulnerabilities, focusing in particular on buffer overflows and heap overflows, with specific attention also to attacks on the Linux kernel.\newline

Chapter 2 will address the concept of buffer overflow, a vulnerability that occurs when a program writes data beyond the buffer boundary, potentially overwriting other portions of memory. We will also explore mitigation techniques, such as ASLR, NX, and PIE, concluding with a practical buffer overflow example.\newline

Chapter 3 will look at heap overflow, a similar vulnerability but involving the heap memory area. We will explore mitigation strategies, including Safe Linking and Top Chunk Integrity Check, along with an illustrative heap overflow example.\newline

Finally, in Chapter 4, we will focus on buffer overflows in the Linux kernel. We will discuss specific vulnerabilities and their mitigations, such as SMEP, SMAP, KASLR, and KPTI, and provide an example of a stack buffer overflow.\newline

This in-depth study of cybersecurity vulnerabilities and their mitigation techniques is critical to understanding and addressing cybersecurity threats in complex environments such as the Linux kernel.\newline

\section{Tooling}
In this thesis, several tools will be utilized, including:\newline


\texttt{pwndbg} : pwndbg is a GDB plugin that makes debugging with GDB with a focus on features needed by low-level software developers, hardware hackers, reverse engineers, and developer exploitation.\newline
    \texttt{pwntools}: is a very powerful Python library created to make difficult things easy in exploit development.\newline
    Such as receiving program output contents, sending user input, sending bytes instead of letters, and much more.\newline
    \texttt{ ghidra or ida free:} 
    ghidra is a free tool developed by the NSA used to decompile binary files.
    ida free is the free version of ida pro and is a cloud base decompiler, the free version works only for some architecture such as x8664.\newline
    Both tools are used in the reverse engineering part.\newline
