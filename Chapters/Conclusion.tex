\chapter{Conclusion}
    In this thesis, various exploitation techniques on different types of memories and environments have been covered, in addition it has been analyzed how mitigations can prevent certain types of attacks from being performed and finally how to bypass these through examples.\newline
    However, my humble opinion is that as time goes on, increasingly powerful mitigations will be implemented and perhaps programming languages will be changed with more secure languages such as rust and many others, making attacks very complicated if not impossible.\newline
    This is not to say that there will no longer be techniques with which future mitigations will be bypassed, but perhaps companies will no longer be interested because too much effort will have to be used to develop.\newline
    The only companies that will be affected will probably be big tech.\newline
    But unfortunately I don't predict the future, so we'll only find out by living.\newline
    \section{online references}
    \href{https://it.wikipedia.org/wiki/Home_page}{wikipedia}\newline
    \href{https://it.wikipedia.org/wiki/Home_page}{Heap Lab course}\newline
    \href{https://www.researchinnovations.com/post/bypassing-the-upcoming-safe-linking-mitigation}{SafeLinkingblog}\newline 
    \href{https://docs.pwntools.com/en/stable/}{Python library pwntools}\newline\
    \href{https://elixir.bootlin.com/}{Linux kernel/Glibc source}\newline
    \href{https://lkmidas.github.io/posts/20210123-linux-kernel-pwn-part-1/}{Linux kernel exploitation blog}