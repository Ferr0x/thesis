\documentclass{report}

\begin{document}

% Pagina di Copertina
\begin{titlepage}
    \centering
    \vspace*{\fill}
    \Huge \textbf{THESIS TITLE}
    \vspace{2cm} 
    
    \Large Lorenzo Ferrari

    \vspace{2cm} % Spazio aggiuntivo tra il nome e il nome dell'istituto
    
    \Large Università degli studi di Parma
    \par
    Three-year degree course in Computer Science
    
 
\end{titlepage}

%indice
\renewcommand{\contentsname}{Index}
\tableofcontents
    
    
    
    %overview 
    \chapter{overview}
    introduction and some stuff
    % Capitolo buffer overflow
    \chapter{Buffer Overflow vulnerability}
    \section{background} %---------------------fix this text ------------------
    Buffer overflow is a critical vulnerability that emerged around the 1970s and 1980s where it was 
    realized through research which leads the attacker to uncontrolled access to critical points of memory.\newline
    As the 1990s arrived the explosion of the Internet and its client-server infrastructure led large numbers of people to use buffer overflow.\newline
    Furthermore, in this period the first books were published explaining how buffer overflow works.\newline
    In the 2000s people wanted to defend themselves from these types of attacks and invented two types of mitigations:\newline
    \begin{itemize}
        \item[$\bullet$] ASLR (Address Space Layout Randomization)
        \item[$\bullet$] stack canary 
    \end{itemize}
    We will explain this mitigation later.\newline
    Between 2000s and 2010s even with the mitigations attackers managed to avoid them and still exploit buffer overflow vulnerability with technique called:\newline
        \begin{itemize}
        \item[$\bullet$] ROP (Return Oriented Programming)
        \item[$\bullet$] RET2LIBC (Return to Libc)
    \end{itemize}
    Even though vulnerability was born so many years ago it still is one of the biggest an dangerous vulnerablity.
    \clearpage
    %-----------------fix until here--------------------
    \section{How Buffer Overflow works}
    A buffer overflow occurs when the attacker can write more input than expected from the buffer, the overflow input exceeds in the memory in the location right after the buffer we are allocating, this could be very dangerous.
    here's an example:
    \begin{verbatim}
    #include <iostream>
    
    int main() {
        char my_input[30]; // buffer victim 
        scanf("%50s",my_input); // wrong usage of scanf  
        return 0;
    }
    \end{verbatim}
    In this example we can see a bad usage of a scanf function, infact this program we have a char buffer with a size of 30 but the scanf function can read until 50 chars 
    \clearpage
    \section{mitigations against Buffer Overflow}
    mitigazioni buffer overflow
    \clearpage
    \section{Buffer Overflow attack and mitigations bypass}
    to be continued
    \clearpage
    \section{explanation of a challange and exploit analyses}
    to be continued
    
    
    %capitolo heap overflow 
    \chapter{Heap Overflow Vulnerability}
    \section{How it works a Heap Overflow}
    to be continued
    \clearpage
    \section{overview challenge and attack planning}
    to be continued
    \clearpage
    \section{exploit analisys}
    to be continued
    \clearpage
    %capitolo double free
    \chapter{Double Free Vulnerability}
    \section{How it works a Double Free}
    to be continued    
    \clearpage
    \section{overview challenge and attack planning}
    to be continued
    \clearpage
    \section{exploit analisys}
    to be continued
    \clearpage
    % capitolo conclusioni
    \chapter{Conclusion}
    to be continued
    \clearpage
\end{document}
